%---------------------------------------------------------------------
%
%                      resumen.tex
%
%---------------------------------------------------------------------
%
% Contiene el cap�tulo del resumen.
%
% Se crea como un cap�tulo sin numeraci�n.
%
%---------------------------------------------------------------------
\begin{comment}
El resumen suele ser la parte del trabajo que primero se lee, por lo que debe ser aut�nomo e identificar el contenido completo del memoria.
Incluir� una breve s�ntesis de cada secci�n del trabajo, desde la introducci�n a las conclusiones. Una de sus funciones es animar al investigador interesado a leer el trabajo completo, por lo que debe reflejar el contenido de forma clara y espec�fica.
Su extensi�n es proporcional a la del trabajo, pero lo habitual es que est� compuesto por un solo p�rrafo de entre 150-250 palabras.
Se redacta en pasado y no debe incluir abreviaturas, referencias a figuras o tablas ni citas bibliogr�ficas. Tampoco se debe incluir informaci�n que no aparezca en el proyecto.
\end{comment}


\chapter{Resumen}
\cabeceraEspecial{Resumen}

Nuestro proyecto consiste en ampliar la herramienta PMCTrack, cuyo fin
es permitir la monitorizaci�n del rendimiento de un programa mediante el uso de
contadores hardware integrados en el procesador.

Esta ampliaci�n est� formada por tres nuevas caracter�sticas.
La primera consiste en modificar la parte interna de PMCTrack para incluir soporte de programas multihilo.
La segunda se trata de implementar una librer�a, *libpmctrack*, que provea de una API para que los desarroladores puedan monitorizar fragmentos de c�digo haciendo uso de los contadores hardware.
La tercera es la construcci�n de una Interfaz Gr�fica de Usuario o \emph{Graphical User
Interface}, que simplifica la configuraci�n de eventos hardware y
permita visualizar gr�ficas de los datos obtenidos en tiempo real.

En este documento, adem�s de la explicaci�n detallada de cada una de estas caracter�sticas, hemos realizado diversos casos de estudio para comprobar y mostrar la utilidad de nuestra aportaci�n a PMCTrack.

\emph{Palabras clave}: , Kernel Linux, Monitorizaci�n rendimiento,

\cabeceraEspecial{Abstract}

\emph{Keywords}: , Linux Kernel, Performance monitoring counters,
Profiling,

\endinput
