%---------------------------------------------------------------------
%
%                      resumen.tex
%
%---------------------------------------------------------------------
%
% Contiene el cap�tulo del resumen.
%
% Se crea como un cap�tulo sin numeraci�n.
%
%---------------------------------------------------------------------
\begin{comment}
El resumen suele ser la parte del trabajo que primero se lee, por lo que debe ser aut�nomo e identificar el contenido completo del memoria.
Incluir� una breve s�ntesis de cada secci�n del trabajo, desde la introducci�n a las conclusiones. Una de sus funciones es animar al investigador interesado a leer el trabajo completo, por lo que debe reflejar el contenido de forma clara y espec�fica.
Su extensi�n es proporcional a la del trabajo, pero lo habitual es que est� compuesto por un solo p�rrafo de entre 150-250 palabras.
Se redacta en pasado y no debe incluir abreviaturas, referencias a figuras o tablas ni citas bibliogr�ficas. Tampoco se debe incluir informaci�n que no aparezca en el proyecto.
\end{comment}


\chapter{Resumen}
\cabeceraEspecial{Resumen}

Nuestro proyecto ha consistido en la ampliaci�n de la herramienta PMCTrack, cuyo fin
es permitir la monitorizaci�n del rendimiento de un programa mediante el uso de los
contadores hardware del procesador.

Esta ampliaci�n ha supuesto la inclusi�n de tres nuevas caracter�sticas. La primera  ha consistido en la modificaci�n de PMCTrack para dar soporte a la monitorizaci�n de programas multihilo desde espacio de usuario. En segundo lugar se ha dotado a PMCTrack de una interfaz de programaci�n para la monitorizaci�n del rendimiento en fragmentos de c�digo espec�ficos. Por �ltimo, se ha procedido al dise�o e implementaci�n de una Interfaz Gr�fica de Usuario o GUI (\emph{Graphical User Interface}), que simplifica la configuraci�n de eventos hardware y permite visualizar gr�ficas de los datos obtenidos en tiempo real.

Para para poner a prueba estas tres nuevas caracter�sticas y mostrar la utilidad de nuestras aportaciones, se han llevado a cabo diversos casos de estudio, los cuales los presentamos tambi�n dentro de este documento.

\emph{Palabras clave}: Monitorizaci�n rendimiento, Monitorizaci�n hardware, Contadores hardware, Kernel Linux, An�lisis c�digo fuente, Aplicaciones multihilo, Monitorizaci�n cach�.

\chapter{Abstract}
\cabeceraEspecial{Abstract}

Our project focused on augmenting the PMCTrack tool for the Linux kernel, whose purpose is to enable monitoring application performance via hardware monitoring counters. The enhancement process entailed the inclusion of three new features in PMCTrack. First, we augmented the tool with support for performance monitoring of multithreaded programs from user space. Second, a programming interface was built on top of PMCTrack's kernel module making it possible to monitor the performance of specific code fragments with hardware counters. Third, we designed and implemented PMCTrack-GUI, a graphical frontend for PMCTrack enabling real-time visualization of high-level performance metrics and specifically designed to simplify the configuration of hardware events to the end user.

To demonstrate the effectiveness of our contributions, we test the functionality of the various PMCTrack extensions carried out in this project by means of several case studies.

\emph{Keywords}: Hardware Profiling, Profiling tools, Performance monitoring counters, Linux kernel, Source code analysis, Multithreaded applications, Cache Monitoring.

\endinput
