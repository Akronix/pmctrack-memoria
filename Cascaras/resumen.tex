%---------------------------------------------------------------------
%
%                      resumen.tex
%
%---------------------------------------------------------------------
%
% Contiene el cap�tulo del resumen.
%
% Se crea como un cap�tulo sin numeraci�n.
%
%---------------------------------------------------------------------
\begin{comment}
El resumen suele ser la parte del trabajo que primero se lee, por lo que debe ser aut�nomo e identificar el contenido completo del memoria.
Incluir� una breve s�ntesis de cada secci�n del trabajo, desde la introducci�n a las conclusiones. Una de sus funciones es animar al investigador interesado a leer el trabajo completo, por lo que debe reflejar el contenido de forma clara y espec�fica.
Su extensi�n es proporcional a la del trabajo, pero lo habitual es que est� compuesto por un solo p�rrafo de entre 150-250 palabras.
Se redacta en pasado y no debe incluir abreviaturas, referencias a figuras o tablas ni citas bibliogr�ficas. Tampoco se debe incluir informaci�n que no aparezca en el proyecto.
\end{comment}


\chapter{Resumen}
\cabeceraEspecial{Resumen}

Nuestro proyecto ha consistido en la ampliaci�n de la herramienta PMCTrack, cuyo fin
es permitir la monitorizaci�n del rendimiento de un programa mediante el uso de los
contadores hardware del procesador.\\
Esta ampliaci�n ha supuesto la inclusi�n de tres nuevas caracter�sticas. La primera  ha consistido en la modificaci�n de PMCTrack para dar soporte a la monitorizaci�n de programas multihilo desde espacio de usuario. En segundo lugar se ha dotado a PMCTrack de una interfaz de programaci�n para la monitorizaci�n del rendimiento en fragmentos de c�digo espec�ficos. Por �ltimo, se ha procedido al dise�o e implementaci�n de una Interfaz Gr�fica de Usuario o GUI (\emph{Graphical User Interface}), que simplifica la configuraci�n de eventos hardware y permite visualizar gr�ficas de los datos obtenidos en tiempo real.



Para para poner a prueba las distintas extensiones de PMCTrack desarrolladas en este proyecto y mostrar la utilidad de nuestras aportaciones se han llevado a cabo diversos casos de estudio. % <--> Adicionalmente, hemos realizado diversos casos de estudio para mostrar el uso y la utilidad de estas extensiones desarrolladas durante el desarrollo de este proyecto, los cuales los presentamos en este documento.

\emph{Palabras clave}: Monitorizaci�n rendimiento, Monitorizaci�n hardware, Contadores hardware, Kernel Linux, An�lisis c�digo fuente, Aplicaciones multihilo, Monitorizaci�n cach�.

\chapter{Abstract}
\cabeceraEspecial{Abstract}

\thispagestyle{empty}

Our project involved extending the PMCTrack tool, whose purpose is to enable performance monitoring of a program by using processor hardware counters.\\
This expansion resulted in the inclusion of three new features. The first involved the modification of PMCTrack to support multithreaded programs monitoring from userspace. Second, a programming interface was provided to PMCTrack for performance monitoring specific code fragments. Finally, we proceeded to design and implement a Graphical User Interface (GUI), which simplifies hardware configuration events and displays monitoring data graphs in real time.

To test the various PMCTrack extensions developed in this project and show the usefulness of our contributions, several case studies were carried out.

\emph{Keywords}: Hardware Profiling, Profiling tools, Performance monitoring counters, Linux kernel, Source code analysis, Multithreaded applications, Cache Monitoring.

\endinput
