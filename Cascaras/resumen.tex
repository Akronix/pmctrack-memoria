%---------------------------------------------------------------------
%
%                      resumen.tex
%
%---------------------------------------------------------------------
%
% Contiene el cap�tulo del resumen.
%
% Se crea como un cap�tulo sin numeraci�n.
%
%---------------------------------------------------------------------
\begin{comment}
El resumen suele ser la parte del trabajo que primero se lee, por lo que debe ser aut�nomo e identificar el contenido completo del memoria.
Incluir� una breve s�ntesis de cada secci�n del trabajo, desde la introducci�n a las conclusiones. Una de sus funciones es animar al investigador interesado a leer el trabajo completo, por lo que debe reflejar el contenido de forma clara y espec�fica.
Su extensi�n es proporcional a la del trabajo, pero lo habitual es que est� compuesto por un solo p�rrafo de entre 150-250 palabras.
Se redacta en pasado y no debe incluir abreviaturas, referencias a figuras o tablas ni citas bibliogr�ficas. Tampoco se debe incluir informaci�n que no aparezca en el proyecto.
\end{comment}


\chapter{Resumen}
\cabeceraEspecial{Resumen}

Nuestro proyecto consiste en ampliar la herramienta PMCtrack, cuyo fin
es monitorizar el rendimiento de un programa mediante el uso de
contadores hardware integrados en el procesador.

Esta ampliaci�n a�ade tres nuevas caracter�sticas: 1. Dotar a la
herramienta de una Interfaz Gr�fica de Usuario o \emph{Graphical User
Interface}, que simplifique la configuraci�n de eventos hardware y
permita visualizar gr�ficas de los datos obtenidos en tiempo real. 2.
Una librer�a, libpmctrack, que provee de una API para monitorizar
fragmentos de c�digo con PMCTrack o para que otros programas hagan uso
de forma sencilla de la monitorizaci�n de contadores integrada en el
kernel. 3. Incluir soporte para programas multihilo.

\emph{Palabras clave}: , Kernel Linux, Monitorizaci�n rendimiento,

\cabeceraEspecial{Abstract}

\emph{Keywords}: , Linux Kernel, Performance monitoring counters,
Profiling,

\endinput
